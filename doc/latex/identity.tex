XML is not just a tree, but a tree with pointers. An xs:IDREF points to an element with an xs:ID within the same document. An xs:ID identifies the immediately containing element. Typically this is an xs:ID attribute on the element, though it may also be xs:ID simple content.

In RDF we identify the element {\itshape content\/}, by assigning it a URI. This URI is the object of the statement representing the element occurrence.


\begin{DoxyItemize}
\item \hyperlink{attributeid}{attribute identity}
\item \hyperlink{elementid}{element identity} 
\end{DoxyItemize}\hypertarget{attributeID}{}\subsection{Attribute Identity}\label{attributeID}
The following example demonstrates the use of xs:ID, xs:IDREF, and xs:IDREFS. The element 'foobar' includes an 'id' attribute that identifies it. The bar element references it via its 'href' or 'hrefs' attributes, of type xs:IDREF or xs:IDREFS, respectively.


\begin{DoxyCodeInclude}
<?xml version="1.0" encoding="UTF-8"?>
<identity xmlns="http://example.org/" >
        <foobar id="foobar"/>
        <bar idref="foobar"/>
        <bar idrefs="foobar foobar" />
</identity>
\end{DoxyCodeInclude}
 
\begin{DoxyCodeInclude}
<?xml version="1.0" encoding="UTF-8"?>
<xs:schema xmlns:xs="http://www.w3.org/2001/XMLSchema" 
        targetNamespace="http://example.org/" xmlns="http://example.org/"
        elementFormDefault="qualified">
        
        <xs:element name="identity">
                <xs:complexType>
                        <xs:sequence>
                                <xs:element name="foobar">
                                        <xs:complexType>
                                                <xs:attribute name="id" type="xs:
      ID"/>
                                        </xs:complexType>
                                </xs:element>
                                <xs:element name="bar" maxOccurs="unbounded">
                                        <xs:complexType>
                                                <xs:attribute name="idref" type="
      xs:IDREF"/>
                                        <xs:attribute name="idrefs" type="xs:IDRE
      FS"/>
                                        </xs:complexType>
                                </xs:element>
                        </xs:sequence>
                </xs:complexType>
        </xs:element>

</xs:schema>
\end{DoxyCodeInclude}


Observe that no resources of types xs:ID, xs:IDREF, or xs:IDREFS appear in the mapped RDF. These are all translated into resource URIs, and in the case of xs:IDREFS to an RDF:List of URIs.


\begin{DoxyCodeInclude}
@prefix ns2:     <http://example.org/def/> .
@prefix ns1:     <http://example.org/> .
@prefix xs_:     <http://www.w3.org/2001/XMLSchema#> .
@prefix xs:      <http://www.w3.org/2001/XMLSchema> .
@prefix rdf:     <http://www.w3.org/1999/02/22-rdf-syntax-ns#> .

<http://example.org/attributeID.xml>
      ns1:identity
              [ ns1:bar [ ns2:idrefs (<http://example.org/attributeID.xml#foobar>
       <http://example.org/attributeID.xml#foobar>)
                        ] ;
                ns1:bar [ ns2:idref <http://example.org/attributeID.xml#foobar>
                        ] ;
                ns1:foobar <http://example.org/attributeID.xml#foobar>
              ] .
\end{DoxyCodeInclude}
 \hypertarget{elementID}{}\subsection{Element Identity}\label{elementID}
This example includes an element 'foobar' with xs:ID content. This identifies the immediately containing element , 'foobar' itself.


\begin{DoxyCodeInclude}
<?xml version="1.0" encoding="UTF-8"?>
<identity xmlns="http://example.org/">
        <foo>foobar</foo>
        <bar>foobar</bar>
</identity>
\end{DoxyCodeInclude}
 
\begin{DoxyCodeInclude}
<?xml version="1.0" encoding="UTF-8"?>
<xs:schema xmlns:xs="http://www.w3.org/2001/XMLSchema" 
        targetNamespace="http://example.org/" elementFormDefault="qualified">
        
        <xs:element name="identity">
                <xs:complexType>
                        <xs:sequence>
                                <xs:element name="foo" type="xs:ID" />
                                <xs:element name="bar" type="xs:IDREF" />
                        </xs:sequence>
                </xs:complexType>
        </xs:element>

</xs:schema>
\end{DoxyCodeInclude}
 
\begin{DoxyCodeInclude}
@prefix ns2:     <http://example.org/> .
@prefix ns1:     <http://example.org/def/> .
@prefix xs_:     <http://www.w3.org/2001/XMLSchema#> .
@prefix rdf:     <http://www.w3.org/1999/02/22-rdf-syntax-ns#> .
@prefix xs:      <http://www.w3.org/2001/XMLSchema> .

<http://example.org/elementIdentity1.xml>
      ns2:identity
              [ ns2:bar <http://example.org/elementIdentity1.xml#foobar> ;
                ns2:foo <http://example.org/elementIdentity1.xml#foobar>
              ] .
\end{DoxyCodeInclude}
 