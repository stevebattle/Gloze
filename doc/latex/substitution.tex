There is a hierarchy among elements defined by substitution groups. These substitution groups define sub-\/property relationships between properties. A substitution group is defined by a head element, and member elements that substitute for the head. The property corresponding to the member is a sub-\/property of that corresponding to the head.

In the example below, the {\itshape head\/} element 'foo' defines the substitution group, of which element 'bar' is a member. This means that in the XML instance, 'bar' may be subsituted for 'foo'. Logically, any RDF statement of 'bar' implies a corresponding statement of 'foo'.


\begin{DoxyCodeInclude}
<?xml version="1.0" encoding="UTF-8"?>
<xs:schema xmlns:xs="http://www.w3.org/2001/XMLSchema" 
        targetNamespace="http://example.org/" xmlns="http://example.org/" >
        
        <xs:element name="foo" type="xs:string" />      
        <xs:element name="bar" substitutionGroup="foo" type="xs:string" />

</xs:schema>
\end{DoxyCodeInclude}
 
\begin{DoxyCodeInclude}
@prefix ns2:     <http://example.org/def/> .
@prefix rdfs:    <http://www.w3.org/2000/01/rdf-schema#> .
@prefix ns1:     <http://example.org/> .
@prefix xs_:     <http://www.w3.org/2001/XMLSchema#> .
@prefix owl:     <http://www.w3.org/2002/07/owl#> .
@prefix xsd:     <http://www.w3.org/2001/XMLSchema#> .
@prefix rdf:     <http://www.w3.org/1999/02/22-rdf-syntax-ns#> .
@prefix xs:      <http://www.w3.org/2001/XMLSchema> .

ns1:bar
      a       rdf:Property , owl:DatatypeProperty ;
      rdfs:range xs_:string ;
      rdfs:subPropertyOf ns1:foo .

ns1:foo
      a       rdf:Property , owl:DatatypeProperty ;
      rdfs:range xs_:string .

<http://example.org/substitution1.owl>
      a       owl:Ontology .
\end{DoxyCodeInclude}
 